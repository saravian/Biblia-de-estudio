\documentclass[twoside,twocolumn,letterpaper,10pt]{memoir}
\usepackage{lipsum}
\usepackage{fixltx2e}
\usepackage{xspace}
\usepackage[usenames,dvipsnames,svgnames,table]{xcolor}
\usepackage{lettrine}
\usepackage{flushend}
\usepackage{fancyhdr}
\usepackage{hyperref}
\usepackage{microtype}
\usepackage[object=vectorian]{pgfornament}
\usepackage[all]{tcolorbox}
\tcbuselibrary{breakable}

\usepackage{fontawesome}
\usepackage{fontawesome5}

\usepackage[T1]{fontenc}
\usepackage{imakeidx}                      % Indice analitico
\makeindex

\pagestyle{fancy}
\fancyhf{}
\fancyhead[L]{\rightmark}
\fancyhead[R]{\leftmark}
\renewcommand{\sectionmark}[1]{\markboth{\thesection. #1}{}}
\renewcommand{\subsectionmark}[1]{\markright{\thesubsection. #1}}
\cfoot{\thepage}
\renewcommand{\headrulewidth}{0pt}

\setcounter{secnumdepth}{5}
\setcounter{tocdepth}{5}

\usepackage{fontspec}
\setmainfont{Montserrat}[
% Files
%  Path      = \string~/s/fonts/equity/ ,
%  Extension = .otf ,
%  % Fonts
%  UprightFont     = Equity Text A Regular ,
%  UprightFeatures = { SmallCapsFont = Equity Caps A Regular } ,
%  BoldFont        = Equity Text A Bold ,
%  BoldFeatures    = { SmallCapsFont = Equity Caps A Bold } ,
%  ItalicFont      = Equity Text A Italic ,
%  BoldItalicFont  = Equity Text A Bold Italic ,
%  % Features
%  Numbers = OldStyle ]
]
% old-style numbers don't look great as drop-caps
\newfontfamily{\lettrinefont}{Comfortaa}[
% Files
%  Path      = \string~/s/fonts/equity/ ,
%  Extension = .otf ,
%  % Fonts
%  UprightFont     = Equity Text A Regular ,
%  UprightFeatures = { SmallCapsFont = Equity Caps A Regular } ,
%  BoldFont        = Equity Text A Bold ,
%  BoldFeatures    = { SmallCapsFont = Equity Caps A Bold } ,
%  ItalicFont      = Equity Text A Italic ,
%  BoldItalicFont  = Equity Text A Bold Italic]
]
\newcommand{\framesize}{\textwidth}
\setlength{\headwidth}{\textwidth}
\setlength{\columnseprule}{0pt}
\setlength{\columnsep}{28pt}
\setlength{\parindent}{0pt}
\setheadfoot{\baselineskip}{2\baselineskip}
\setheaderspaces{*}{12pt}{*}
\clubpenalty10000
\widowpenalty10000

\usepackage[british]{babel}

\newlength{\versespacing}
\setlength{\versespacing}{.16667em plus .08333em}
\newcommand{\versespace}{\hspace{\versespacing}}
\frenchspacing

\usepackage{eso-pic}
\newcommand\AtPageUpperRight[1]{\AtPageUpperLeft{%
		\put(\LenToUnit{\paperwidth},\LenToUnit{0\paperheight}){#1}%
}}%
\newcommand\AtPageLowerRight[1]{\AtPageLowerLeft{%
		\put(\LenToUnit{\paperwidth},\LenToUnit{0\paperheight}){#1}%
}}%

\newcommand{\sectionstyle}{\bfseries\raggedright\Large}
\setsecheadstyle{\sectionstyle}
\setbeforesecskip{0ex}
\setaftersecskip{0.1ex}

\newcommand{\subsectionstyle}{\bfseries\itshape\raggedright\large}
\setsubsecheadstyle{\subsectionstyle}
\setbeforesubsecskip{0ex}
\setaftersubsecskip{0.1ex}

\makeatletter
\newcommand\versenumcolor{red}
\newcommand\chapnumcolor{red}
\newlength{\biblechapskip}
\setlength{\biblechapskip}{1em plus .33em minus .2em}
\newcounter{biblechapter}
\newcounter{bibleverse}[biblechapter]
\renewcommand\chaptername{Book}
\newcommand{\biblebook}[1]{%
	\setcounter{biblechapter}{0}
	\gdef\currbook{#1}
	\chapter*{#1}
	\addcontentsline{toc}{chapter}{#1}}
\newcount\biblechap@svdopt
\newenvironment{biblechapter}[1][\thebiblechapter]
{\biblechap@svdopt=#1
	\ifnum\c@biblechapter=\biblechap@svdopt\else
	\advance\biblechap@svdopt by -1\fi
	\setcounter{biblechapter}{\the\biblechap@svdopt}
	\stepcounter{biblechapter}
	\setbeforesecskip{1ex}\setbeforesubsecskip{1ex}
	\lettrine[lines=3,lhang=0,findent=0pt,nindent=0pt,loversize=0.25]{\lettrinefont\color{\chapnumcolor}\,\thebiblechapter\,}{}\ignorespaces}
{\par\vspace{\biblechapskip}\setbeforesecskip{0ex}\setbeforesubsecskip{0ex}}
\newcommand{\@showversenum}{%
	\ifnum\c@bibleverse=1\else{\color{\versenumcolor}\textbf{\thebibleverse}~}\fi%
	\markboth{{\scshape\currbook} \thebiblechapter:\thebibleverse}{{\scshape\currbook} \thebiblechapter:\thebibleverse}%
	\ignorespaces}
\renewcommand{\verse}{\stepcounter{bibleverse}\@showversenum}
\newcommand{\verseWithHeading}[1]{%
	\stepcounter{bibleverse}%
	\ifnum\c@bibleverse=1{\sectionstyle#1\newline}\else\section*{#1}\fi\@showversenum}
\newcommand{\verseWithSubheading}[1]{%
	\stepcounter{bibleverse}%
	\ifnum\c@bibleverse=1{\subsectionstyle#1\newline}\else\subsection*{#1}\fi\@showversenum}
\makeatother

\newcommand{\startornaments}{\AddToShipoutPictureBG{%
		\checkoddpage%
		\ifoddpage%
		\AtPageUpperRight{\put(-60,-35){\pgfornament[width=1.75cm,symmetry=v,color=Goldenrod]{63}}}%
		\AtPageLowerRight{\put(-60,35){\pgfornament[width=1.75cm,symmetry=c,color=Goldenrod]{63}}}%
		\else%
		\AtPageUpperLeft{\put(10,-35){\pgfornament[width=1.75cm,color=Goldenrod]{63}}}%
		\AtPageLowerLeft{\put(10,35){\pgfornament[width=1.75cm,symmetry=h,color=Goldenrod]{63}}}
		\fi}}

\newcommand{\stopornaments}{\ClearShipoutPictureBG}

\renewcommand{\printparttitle}[1]{%
	\thispagestyle{empty}%
	\addcontentsline{toc}{part}{#1}%
	\vspace*{\fill}%
	\begin{tikzpicture}[transform shape,every node/.style={inner sep=0pt}]%
		\node[minimum size=\framesize](vecbox){};%
		\node[inner sep=6pt, color=black] (text) at (vecbox.center){%
			\HUGE \textsc{#1}};%
		\node[anchor=north, color=Goldenrod] (base) at (text.south){%
			\pgfornament[width=0.5*\framesize]{88}};%
	\end{tikzpicture}%
	\vspace*{\fill}}

\makechapterstyle{dash-embiggened}{%
	\chapterstyle{default}
	\setlength{\beforechapskip}{5\onelineskip}
	\renewcommand*{\printchaptername}{}
	\renewcommand*{\chapternamenum}{}
	\renewcommand*{\chapnumfont}{\normalfont\Huge}
	\settoheight{\midchapskip}{\chapnumfont 1}
	\renewcommand*{\printchapternum}{\centering \chapnumfont
		\rule[0.5\midchapskip]{1em}{0.4pt} \thechapter\
		\rule[0.5\midchapskip]{1em}{0.4pt}}
	\renewcommand*{\afterchapternum}{\par\nobreak\vskip 0.5\onelineskip}
	\renewcommand*{\printchapternonum}{\centering
		\vphantom{\chapnumfont 1}\afterchapternum}
	\renewcommand*{\chaptitlefont}{\normalfont\HUGE\scshape}
	\renewcommand*{\printchaptertitle}[1]{\centering \chaptitlefont ##1}
	\setlength{\afterchapskip}{2.5\onelineskip}}

\chapterstyle{dash-embiggened}

\newcommand{\columnbreak}{\pagebreak}

\newcommand{\LORD}{\textsc{Lord}\xspace}
\newcommand{\LORDs}{\textsc{Lord's}\xspace}


\definecolor{rosa}{RGB}{255,212,229}
\definecolor{violeta}{RGB}{224,205,255}
\definecolor{celeste}{RGB}{189,232,239}
\definecolor{verde}{RGB}{183,215,132}
\definecolor{amarillo}{RGB}{254,255,162}

\usepackage{soul}
\newcommand{\motivo}[1]{def}
\newcommand{\creacion}[1]{def}
\newcommand{\reino}[1]{\faCrown \space sethlcolor{celeste}\hl{#1}\index{#1}}
\newcommand{\pacto}[1]{\faHandshake \space \sethlcolor{rosa}\hl{#1}\index{#1}}
\newcommand{\dispensacion}[1]{\faCrown \space \sethlcolor{violeta}\hl{#1}\index{#1}}
\newcommand{\epoca}[1]{\faClock \space \sethlcolor{verde}\hl{#1}\index{#1}}
\newcommand{\juicio}[1]{\faBalanceScaleLeft \space \sethlcolor{amarillo}\hl{#1}\index{#1}}
\newcommand{\promesa}[1]{\faGifts \space \setulcolor{yellow}\ul{#1}\index{#1}}